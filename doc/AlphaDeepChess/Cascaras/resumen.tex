\chapter*{Resumen}

\section*{\tituloPortadaVal}

Los motores de ajedrez han influido notablemente en el desarrollo de estrategias computacionales y algoritmos de juego desde mediados del siglo XX. Informáticos de la talla de Alan Turing y Claude Shannon sentaron las bases para el desarrollo de este campo. Posteriormente, las mejoras de hardware y software y la evolución de la heurística se asentarían sobre estos cimientos, incluida la introducción de la poda alfa-beta, una optimización del algoritmo minimax que reducía significativamente el número de nodos evaluados en un árbol de juego. Con el aumento de la potencia de cálculo, motores modernos como Stockfish o Komodo aprovechan no sólo las optimizaciones de búsqueda, sino también los avances en heurística y, en algunos casos, la inteligencia artificial mediante redes neuronales.

\section*{Palabras clave}
   
\noindent motor de ajedrez,poda alfa-beta,algoritmo minimax,búsqueda árbol de juego,evaluación heurística,ordenación de jugadas,optimización de búsqueda,tablas de transposición,zobrist hashing

   


