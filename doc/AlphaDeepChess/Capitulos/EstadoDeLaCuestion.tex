\chapter{State of the art}
\label{cap:estadoDeLaCuestion}

\section{Board representation}

Let's introduce the battlefield where the game takes place: the chessboard, which serves as the foundation for all operations.

\vspace{1em}

In order to store a position setting with its pieces and its piece colors and other additional information like the side to move\footnote{Corresponds to the colour that has the turn of movement.}, the castling rights\footnote{Refers to the possibilities for each side to castle both short and long. Castling is explained in \ref{sec:castling}.} or the fifty-move rule counter\footnote{This rule is explained in \ref{itm:fifty-move-rule}.}, we can encounter different types of representations: piece centric, square centric and hybrid solutions. We chose to use bitboards as the primary representation, complemented by a piece list to store the piece on each square (piece centric representation). Additionally, the game state is stored in a bit field.

\vspace{1em}

A \textbf{bitboard\footnote{\url{https://www.chessprogramming.org/Board_Representation}}}, also known as a bitset or bitmap, is a 64-bit word structure that efficiently represents a  chessboard because it matches with the number of squares: every bit corresponds to a square of the chessboard.

\vspace{1em}

A \textbf{bit field}, in constrast to a bitboard, uses a fixed number of bits within an integer to store multiple small values or flags.

\vspace{1em}

For example, using a 64-bit bitfield, we can allocate 6 bits to store the number of pieces on the board, as $2^6 = 64$ possibilities. This means the bits would occupy an interval from $X$ included to $X+5$.

\vspace{1em}

For more information about board representation, refer to the Wikipedia page: \url{https://www.chessprogramming.org/Board_Representation}

\section{Game trees}

Sequential games, such as chess or tic-tac-toe, where players take turns alternately, unlike simultaneous games, can be represented in a game tree or graph. In this representation, each node is a possible option or game state, and child nodes branch out to represent subsequent states, forming a tree-like structure. This tree has a height or depth that refers to the number of levels or layers in the tree, starting from the root node (the initial game state) and extending to the leaf nodes.

\vspace{1em}

The depth of a chess game tree is important because it determines the extent to which it will be analysed and evaluated. A depth of 1 represents all possible moves for the current player or side to move, while a depth of 2 includes the opponent's responses to those moves. As the depth increases, the tree grows exponentially, making it computationally expensive to explore all possible states.

\vspace{1em}

The search algorithms in these game trees will be discussed below.

\section{Search algorithms}

\subsection{Minimax algorithm}

\subsubsection{Alpha-beta pruning}

\section{Move generation}

\section{Evaluation functions}

\section{Move ordering}

\section{Iterative deepening}

\section{Quiescence search}

\section{Transposition tables}

\section{Zobrist hashing}

\section{Parallel search}

\section{Strength Assessment}

%En el estado de la cuestión es donde aparecen gran parte de las referencias bibliográficas del trabajo. Una de las formas más cómodas de gestionar la bibliografía en {\LaTeX} es utilizando \textbf{bibtex}. Las entradas bibliográficas deben estar en un fichero con extensión \textit{.bib} (con esta plantilla se proporciona el fichero biblio.bib, donde están las entradas referenciadas más abajo). Cada entrada bibliográfica tiene una clave que permite referenciarla desde cualquier parte del texto con los siguiente comandos:

%\begin{itemize}
%\item Referencia bibliografica con cite: \cite{ldesc2e}
%\item Referencia bibliográfica con citep: \citep{notsoshort}
%\item Referencia bibliográfica con citet: \citet{latexAPrimer}
%\end{itemize}

%Es posible citar más de una fuente, como por ejemplo \citep{latexCompanion,LaTeXLamport,texKnuth}

%Después, \LaTeX se ocupa de rellenar la sección de bibliografía con las entradas \textbf{que hayan sido citadas} (es decir, no con todas las entradas que hay en el .bib, sino sólo con aquellas que se hayan citado en alguna parte del texto).

%Bibtex es un programa separado de latex, pdflatex o cualquier otra cosa que se use para compilar los .tex, de manera que para que se rellene correctamente la sección de bibliografía es necesario compilar primero el trabajo (a veces es necesario compilarlo dos veces), compilar después con bibtex, y volver a compilar otra vez el trabajo (de nuevo, puede ser necesario compilarlo dos veces). 
