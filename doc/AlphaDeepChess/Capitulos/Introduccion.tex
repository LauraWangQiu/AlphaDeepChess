\chapter{Introduction}
\label{cap:introduction}
\renewcommand{\figurename}{Figure}

\chapterquote{The most powerful weapon in chess is to have the next move}{David Bronstein}

Chess, one of the oldest and most strategic games in human history, has long been a domain for both intellectual competition and computational research. The pursuit of creating a machine that could compete with the best human players, chess Grandmasters (GM), was present. It was only a matter of time before computation surpassed human computational capabilities.

\vspace{1em}

\noindent In 1997, the chess engine Deep Blue made history by defeating the world champion at the time, Garry Kasparov, marking the first time a computer had defeated a reigning world champion in a six-game match under standard chess tournament time controls\footnote{\url{https://en.wikipedia.org/wiki/Deep_Blue_(chess_computer)}}.

\vspace{1em}

\noindent Since then, the development of chess engines has advanced rapidly, moving from rule-based systems to AI-driven models. However, classical search algorithms, such as alpha-beta pruning, continue to be fundamental to understanding the basics of efficient search and evaluation of game trees.

%\section{Motivation}

\section{Where to begin?}

Let's start from the beginning. What is chess? Chess is a board game where two players who take white pieces and black pieces respectively compete to first checkmate\footnote{\url{https://en.wikipedia.org/wiki/Checkmate}} the opponent.

\vspace{1em}

\noindent What about a chess engine? A chess engine consists of a software program that analyzes chess positions and returns optimal moves depending on its configuration. In order to help users to use these engines, chess community agreed on creating an open communication protocol called \textbf{Universal Chess Interface} or commonly referred to as UCI, that provides the interaction with chess engines through user interfaces.

\vspace{1em}

\noindent In the following section~\ref{sec:basicConcepts}, we will talk about the basic concepts of chess, but if you already have the knowledge we recommend you to advance directly to the chapter~\ref{cap:estadoDeLaCuestion}.

\section{Basic concepts}
\label{sec:basicConcepts}

Chess is a game of strategy that takes place on a chessboard with specific rules governing the movement and interaction of the pieces. This section introduces the fundamental concepts necessary to understand how chess is played.

\subsection{Chessboard}
\label{sec:chessboard}

A chessboard is a game board of 64 squares, 8 rows by 8 columns. To refer to each of the squares we mostly use \textbf{algebraic notation\footnote{\url{https://en.wikipedia.org/wiki/Algebraic_notation_(chess)}}} using the numbers from 1 to 8 and the letters from ``a'' to ``h''. There are also other notations like descriptive notation\footnote{\url{https://en.wikipedia.org/wiki/Descriptive_notation}} which is obsolete or ICCF numeric notation\footnote{\url{https://en.wikipedia.org/wiki/ICCF_numeric_notation}} due to chess pieces have different abbreviations depending on language.

\begin{figure}[H] % H para que la figura se coloque exactamente en la posición donde esté definida (del paquete float)
    \centering
    \newchessgame
    \chessboard[setpieces={},showmover=false]
    \caption{Empty chessboard.}
    \label{fig:chessboard}
\end{figure}

\newpage 

\noindent For example, $g5$ refers to the following square:

\begin{figure}[H]
    \centering
    \newchessgame % Crear un nuevo tablero vacío
    \chessboard[
      setpieces={}, % No piezas en el tablero
      showmover=false,
      markstyle=circle, color=red, markfield=g5, % Marcar g5 con un círculo rojo
      pgfstyle=straightmove, color=blue, % Estilo de flechas rectas y azules
      markmoves={g1-g5, a5-g5}, % Flechas desde g1 → g5 y a5 → g5
      arrow=to % Flechas con punta
    ]
    \caption{Example: square $g5$ highlighted and arrows pointing to it.}
\end{figure}

\noindent It is important to know that when placing a chessboard in the correct orientation, there should always be a \textbf{white square in the bottom-right corner} or a \textbf{black square in the bottom-left corner}.

\subsection{Chess pieces}

There are 6 types of chess pieces: king, queen, rook, bishop, knight and pawn, and each side has 16 pieces:

\begin{table}[H]
    \centering
    \resizebox{\textwidth}{!}{
    \begin{tabular}{|c|c|c|c|}
        \hline
        \textbf{Piece} & \textbf{White Pieces} & \textbf{Black Pieces} & \textbf{Number of Pieces} \\ \hline
        King           & \WhiteKingOnWhite     & \BlackKingOnWhite     & 1                         \\ \hline
        Queen          & \WhiteQueenOnWhite    & \BlackQueenOnWhite    & 1                         \\ \hline
        Rook           & \WhiteRookOnWhite\WhiteRookOnWhite & \BlackRookOnWhite\BlackRookOnWhite & 2 \\ \hline
        Bishop         & \WhiteBishopOnWhite\WhiteBishopOnWhite & \BlackBishopOnWhite\BlackBishopOnWhite & 2 \\ \hline
        Knight         & \WhiteKnightOnWhite\WhiteKnightOnWhite & \BlackKnightOnWhite\BlackKnightOnWhite & 2 \\ \hline
        Pawn           & \WhitePawnOnWhite\WhitePawnOnWhite\WhitePawnOnWhite\WhitePawnOnWhite\WhitePawnOnWhite\WhitePawnOnWhite\WhitePawnOnWhite\WhitePawnOnWhite & \BlackPawnOnWhite\BlackPawnOnWhite\BlackPawnOnWhite\BlackPawnOnWhite\BlackPawnOnWhite\BlackPawnOnWhite\BlackPawnOnWhite\BlackPawnOnWhite & 8 \\ \hline
    \end{tabular}
    }
    \caption{Number of chess pieces by type and color.}
    \label{tab:number-of-pieces}
\end{table}

\noindent The starting position of the chess pieces on a chessboard is the following:

\begin{figure}[H]
    \centering
    \newchessgame
    \chessboard[showmover=false]
    \caption{Starting position.}
    \label{fig:start-pos}
\end{figure}

\noindent Notice that the queen and king are placed in the center columns. The queen is placed on a square of its color, while the king is placed on the remaining central column. The rest of the pieces are positioned symmetrically, as shown in Figure \ref{fig:start-pos}. This means that the chessboard is divided into two sides relative to the positions of the king and queen at the start of the game:

\begin{figure}[H]
  \centering
  \newchessgame
  \chessboard[
    showmover=false,
    setpieces={},
    pgfstyle=color, opacity=0.2,
    color=red,
    markfields={a1,a2,a3,a4,a5,a6,a7,a8,b1,b2,b3,b4,b5,b6,b7,b8,c1,c2,c3,c4,c5,c6,c7,c8,d1,d2,d3,d4,d5,d6,d7,d8},
    color=blue!50,
    markfields={e1,e2,e3,e4,e5,e6,e7,e8,f1,f2,f3,f4,f5,f6,f7,f8,g1,g2,g3,g4,g5,g6,g7,g8,h1,h2,h3,h4,h5,h6,h7,h8}
  ]
  \caption{King's side (blue) and Queen's side (red).}
  \label{fig:kings-queens-side}
\end{figure}

\subsection{Movement of the pieces}
\label{sec:movement-pieces}

\subsubsection{Pawn}

The pawn can move one square forward, but it can only capture pieces one square diagonally. On its first move, the pawn has the option to move two squares forward. If a pawn reaches the last row of the opponent's side, it can be promoted\footnote{\url{https://en.wikipedia.org/wiki/Promotion_(chess)}} to any other piece (except for a king).

\begin{figure}[H]
    \centering
    \begin{minipage}{0.45\textwidth}
        \centering
        \newchessgame
        \chessboard[
            setpieces={Pe2, Pc4, pc5, Pg3, pf6},
            showmover=false,
            pgfstyle=straightmove, color=blue,
            markmoves={e2-e3,e2-e4,g3-g4,f6-f5},
            arrow=to
        ]
        \caption{Pawn's movement.}
        \label{fig:pawn-movement}
    \end{minipage}
    \begin{minipage}{0.45\textwidth}
        \centering
        \newchessgame
        \chessboard[
           setpieces={Pe2, Pc4, pc5, Pg3, pf6},
           showmover=false,
           pgfstyle=straightmove, color=red,
           markmoves={e2-d3,e2-f3,c4-b5,c4-d5,g3-f4,g3-h4,f6-e5,f6-g5,c5-b4,c5-d4},
           arrow=to
       ]
       \caption{Pawn attack.}
       \label{fig:pawn-attack}
    \end{minipage}
\end{figure}

\begin{figure}[H]
    \centering
    \begin{minipage}{0.45\textwidth}
        \centering
        \newchessgame
        \chessboard[
            setpieces={Pe7},
            showmover=false,
            pgfstyle=straightmove, color=blue,
            markmoves={e7-e8},
            arrow=to
        ]
        \caption{Promotion.}
        \label{fig:pawn-promotion}
    \end{minipage}
    \begin{minipage}{0.45\textwidth}
        \centering
        \newchessgame
        \chessboard[
            setpieces={Qe8},
            showmover=false
        ]
        \caption{Pawn promotes to queen.}
        \label{fig:pawn-promotion-2}
    \end{minipage}
\end{figure}

\noindent There is a specific capture movement which is \textbf{en passant \footnote{\url{https://en.wikipedia.org/wiki/En_passant}}}. This move allows a pawn that has moved two squares forward from its starting position to be captured by an opponent's pawn as if it had only moved one square. The capturing pawn must be on an adjacent file and can only capture the en passant pawn immediately after it moves.

\begin{figure}[H]
    \centering
    \begin{minipage}[H]{0.3\textwidth}
        \centering
        \setchessboard{boardfontsize=15pt,labelfontsize=10pt}
        \newchessgame[black]
        \chessboard[
            setpieces={ke8,Ke1,pd7,Pe5},
            showmover=false,
            pgfstyle=straightmove, color=blue,
            markmoves={d7-d5},
            arrow=to
        ]
        \caption{\centering En passant (1).}
        \label{fig:en-passant-1}
    \end{minipage}
    \hfill
    \begin{minipage}[H]{0.3\textwidth}
        \centering
        \setchessboard{boardfontsize=15pt,labelfontsize=10pt}
        \newchessgame
        \chessboard[
            setpieces={ke8,Ke1,pd5,Pe5},
            showmover=false,
            pgfstyle=straightmove, color=red,
            markmoves={e5-d6},
            arrow=to
        ]
        \caption{\centering En passant (2).}
        \label{fig:en-passant-2}
    \end{minipage}
    \hfill
    \begin{minipage}[H]{0.3\textwidth}
        \centering
        \setchessboard{boardfontsize=15pt,labelfontsize=10pt}
        \newchessgame
        \chessboard[
            setpieces={ke8,Ke1,Pd6},
            showmover=false
        ]
        \caption{\centering En passant (3).}
        \label{fig:en-passant-3}
    \end{minipage}
\end{figure}

\subsubsection{Rook}

The rook can move any number of squares horizontally or vertically. It can also capture pieces in the same way.

\begin{figure}[H]
    \centering
    \newchessgame
    \chessboard[
        setpieces={Rd4,Rg6,Ng2,bc6},
        showmover=false,
        pgfstyle=straightmove, color=blue,
        markmoves={d4-a4,d4-h4,d4-d1,d4-d8,g6-h6,g6-g8,g6-c6,g6-g3},
        arrow=to
    ]
    \caption{Rook's movement.}
    \label{fig:rook-movement}
\end{figure}

\subsubsection{Knight}

The knight moves in an L-shape: two squares in one direction and then one square perpendicular to that direction. The knight can jump over other pieces, making it a unique piece in terms of movement. It can also capture pieces in the same way.

\begin{figure}[H]
    \centering
    \newchessgame
    \chessboard[
        setpieces={Nf3,Na8,pb7},
        showmover=false,
        pgfstyle=straightmove, color=blue,
        markmoves={f3-e5,f3-e1,f3-g5,f3-g1,f3-d2,f3-d4,f3-h2,f3-h4,a8-b6,a8-c7},
        arrow=to
    ]
    \caption{Knight's movement.}
    \label{fig:knight-movement}
\end{figure}

\subsubsection{Bishop}

The bishop can move any number of squares diagonally. It can also capture pieces in the same way. Considering that each side has two bishops, one bishop moves on light squares and the other on dark squares.

\begin{figure}[H]
    \centering
    \newchessgame
    \chessboard[
        setpieces={Bc4,Bf6,nb2,Qe7},
        showmover=false,
        pgfstyle=straightmove, color=blue,
        markmoves={c4-a2,c4-g8,c4-f1,c4-a6,f6-h4,f6-h8,f6-b2},
        arrow=to
    ]
    \caption{Bishop's movement.}
    \label{fig:bishop-movement}
\end{figure}

\subsubsection{King}

The king can move one square in any direction: horizontally, vertically, or diagonally. However, the king cannot move to a square that is under attack by an opponent's piece. The king can also capture pieces in the same way. The king is a crucial piece in chess, as the game ends when one player checkmates the opponent's king.

\begin{figure}[H]
    \centering
    \begin{minipage}{0.45\textwidth}
       \centering
       \newchessgame
       \chessboard[
           setpieces={Ke4},
           showmover=false,
           pgfstyle=straightmove, color=blue,
           markmoves={e4-e5,e4-e3,e4-d4,e4-f4,e4-d5,e4-f5,e4-d3,e4-f3},
           arrow=to
       ]
       \caption{King's movement.}
       \label{fig:king-movement}
   \end{minipage}
   \begin{minipage}{0.45\textwidth}
       \centering
       \newchessgame
       \fenboard{8/8/1p6/p1pBk1p1/P1P2p1p/1P1K1P1P/1b4P1/8 w - - 0 1}
       \chessboard[
           markstyle=circle, color=red, markfields={d4,c3,e4,e3},
           pgfstyle=straightmove, color=blue,
           markmoves={d3-c2,d3-d2,d3-e2},
           arrow=to
       ]
       \caption{White King's movement in a game.}
       \label{fig:white-king-movement-game}
   \end{minipage}
\end{figure}

\noindent In Figure \ref{fig:white-king-movement-game}, the white king cannot move to $e4$ because the black king is attacking that square. When two kings are positioned close to each other, neither can move to a square adjacent to the other.

\vspace{1em} % Añade espacio

\noindent Additionally, the king can perform a special move called \textbf{castling}, which involves moving the king two squares towards a rook and moving the rook to the square next to the king. Castling has specific conditions which are:

\begin{itemize}
    \item Neither the king nor the rook involved in castling must have moved previously.
    \item There must be no pieces between the king and the rook.
    \item The king cannot be in check, move through a square under attack, or end up in check.
\end{itemize}

\begin{figure}[H]
    \centering
    \newchessgame
    \chessboard[
        setpieces={Ke1,Ra1,Rh1,ke8,rb8,rh8,bf8},
        showmover=false,
        pgfstyle=straightmove, color=blue,
        markmoves={e1-c1, e1-g1},
        arrow=to
    ]
    \caption{Castling}
    \label{fig:castling}
\end{figure}

\noindent In this case, the white king can castle on either the king's side or the queen's side as long as the rooks have not been moved from their starting position, but the black king cannot castle because there is a bishop on $f8$ interfering with the movement and the rook on the queen's side has been moved to $b8$.

\subsubsection{Queen}

The queen can move any number of squares in any direction: horizontally, vertically, or diagonally. It can also capture pieces in the same way.

\begin{figure}[H]
    \centering
    \newchessgame
    \chessboard[
        setpieces={Qd4},
        showmover=false,
        pgfstyle=straightmove, color=blue,
        markmoves={d4-a4,d4-h4,d4-d1,d4-d8,d4-a1,d4-h1,d4-h8,d4-a8},
        arrow=to
    ]
    \caption{Queen's movement.}
    \label{fig:queen-movement}
\end{figure}

\subsection{Rules}
\label{sec:rules}

Each player aims to checkmate the opponent's king, which means that the king is under attack and cannot escape.

\vspace{1em}

\noindent In every game, \textbf{white starts first} and the possible results of each game can be win for white, win for black or draw\footnote{\url{https://en.wikipedia.org/wiki/Draw_(chess)}}. A tie could be caused by different conditions:

\begin{enumerate}
    \item Stalemate: the player whose turn it is to move has no legal moves, and their king is not in check.
    \item Insufficient material: neither player has enough pieces to checkmate. Those cases are king vs king, king and bishop vs king, king and knight vs king, and king and bishop vs king and bishop with the bishops on the same color.
    \item Threefold repetition: it occurs when same position happens three times during the game, with the same player to move and the same possible moves (including castling and en passant).
    \item Fifty-move rule: if 50 consecutive moves are made by both players without a pawn move or a capture, the game can be declared a draw. \label{itm:fifty-move-rule}
    \item Mutual agreement: both players can agree to a draw at any point during the game.
    \item Dead position: a position where no legal moves can be made, and the game cannot continue. This includes cases like king vs king, king and knight vs king, or king and bishop vs king.
\end{enumerate}

\begin{figure}[H]
    \centering
    \begin{minipage}[H]{0.3\textwidth}
        \centering
        \setchessboard{boardfontsize=15pt,labelfontsize=10pt}
        \newchessgame
        \chessboard[
            setfen={2k5/8/8/3QB3/8/4K3/8/8 b - - 0 1},
            markstyle=circle, color=red, markfields={b8,b7,c7,d7,d8},
            pgfstyle=straightmove, color=blue,
            markmoves={d5-d8,d5-a8,e5-b8},
            arrow=to
        ]
        \caption{\centering Stalemate.}
        \label{fig:stalemate}
    \end{minipage}
    \hfill
    \begin{minipage}[H]{0.3\textwidth}
        \centering
        \setchessboard{boardfontsize=15pt,labelfontsize=10pt}
        \newchessgame
        \chessboard[
            setfen={8/8/8/4k3/4B3/4K3/8/8 w - - 0 1}
        ]
        \caption{\centering Insufficient material.}
        \label{fig:insufficient-material}
    \end{minipage}
    \hfill
    \begin{minipage}[H]{0.3\textwidth}
        \centering
        \setchessboard{boardfontsize=15pt,labelfontsize=10pt}
        \newchessgame
        \chessboard[
            setfen={8/2b1k3/7p/p1p1p1pP/PpP1P1P1/1P1BK3/8/8 w - - 0 1}
        ]
        \caption{\centering Dead position.}
        \label{fig:dead-position}
   \end{minipage}
\end{figure}

\noindent Players can also resign at any time, conceding victory to the opponent. Also, if a player runs out of time in a timed game, they lose unless the opponent does not have enough material to checkmate, in which case the game is drawn.

\vspace{1em}

\noindent For more information about chess rules, refer to the Wikipedia page: \url{https://en.wikipedia.org/wiki/Rules_of_chess}.

\newpage

\subsection{Notation}

Notation is important in chess to record moves and analyze games.

\subsubsection{Algebraic notation}

In addition to the \textbf{algebraic notation} of the squares in section \ref{sec:chessboard}, each piece is identified by an uppercase letter, which may vary across different languages:

\begin{table}[H]
    \centering
    \begin{tabular}{|c|c|c|}
        \hline
        \textbf{Piece} & \textbf{English Notation} & \textbf{Spanish Notation} \\ \hline
        Pawn           & \textit{P}               & \textit{P} (peón)         \\ \hline
        Rook           & \textit{R}               & \textit{T} (torre)        \\ \hline
        Knight         & \textit{N}               & \textit{C} (caballo)      \\ \hline
        Bishop         & \textit{B}               & \textit{A} (alfil)        \\ \hline
        Queen          & \textit{Q}               & \textit{D} (dama)         \\ \hline
        King           & \textit{K}               & \textit{R} (rey)          \\ \hline
    \end{tabular}
    \caption{Chess piece notation in English and Spanish.}
    \label{tab:chess-notation}
\end{table}

\paragraph{Normal moves (not captures nor promoting)}
It is written using the piece uppercase letter plus the coordinate of destination:

\begin{figure}[H]
    \centering
    \newchessgame
    \chessboard[
        setfen={r1bqkbnr/pppp1ppp/2n5/1B2p3/4P3/5N2/PPPP1PPP/RNBQKB1R b KQkq - 0 1},
        pgfstyle=straightmove, color=blue,
        markmoves={a7-a6},
        arrow=to
    ]
    \caption{Pawn goes to a6.}
    \label{fig:pawn-a6}
\end{figure}

In Figure \ref{fig:pawn-a6}, the pawn's movement is written as $Pa6$ or directly as $a6$.

\paragraph{Captures}
They are written with an $"x"$ between the piece uppercase letter and coordinate of destination/the captured piece coordinate:

\begin{figure}[H]
    \centering
    \newchessgame
    \chessboard[
        setfen={r1bqkbnr/1ppp1ppp/p1n5/1B2p3/4P3/5N2/PPPP1PPP/RNBQKB1R w KQkq - 0 1},
        pgfstyle=straightmove, color=red,
        markmoves={b5-c6},
        arrow=to
    ]
    \caption{Bishop captures knight.}
    \label{fig:bishop-captures-knight}
\end{figure}

\noindent In Figure \ref{fig:bishop-captures-knight}, the white bishop capturing the black knight is written as $Bxc6$. If it were black's turn, the pawn on $a6$ could capture the white bishop, and it would be written as $Pxb5$ or simply $axb5$, indicating the pawn's column.

\paragraph{Pawn promotion}
It is written as the pawn's movement to the last row, followed by the piece to which it is promoted:

\begin{figure}[H]
    \centering
    \newchessgame
    \chessboard[
        setfen={r7/1Pp5/2P3p1/8/6pb/4p1kB/4P1p1/6K1 w - - 0 1},
        pgfstyle=straightmove, color=blue,
        markmoves={b7-a8},
        arrow=to
    ]
    \caption{Pawn captures rook.}
    \label{fig:pawn-captures-rook}
\end{figure}

\noindent In Figure \ref{fig:pawn-captures-rook}, white pawn capturing and promoting in $a8$ to a queen is written as $bxa8Q$ or $bxa8=Q$.

\paragraph{Castling}
\label{sec:castling}
Depending on whether it is on the king's side or the queen's side, it is written as $0-0$ and $0-0-0$, respectively.

\paragraph{Check and checkmate}
They are written by adding a $+$ sign for check or $++$ for checkmate, respectively.

\begin{figure}[H]
    \centering
    \newchessgame
    \chessboard[
        setfen={rnb1kbnr/pppp1ppp/8/4p3/6Pq/5P2/PPPPP2P/RNBQKBNR w KQkq - 0 1},
        pgfstyle=straightmove, color=blue,
        markmoves={d8-h4},
        arrow=to
    ]
    \caption{Black queen checkmates.}
    \label{fig:black-queen-checkmates}
\end{figure}

\noindent In Figure \ref{fig:black-queen-checkmates}, black queen movement checkmates and it is written as $Dh4++$.

\paragraph{The end of game notation} indicates the result of the game. It is typically written as:

\begin{itemize}
    \item \textbf{1-0}: White wins.
    \item \textbf{0-1}: Black wins.
    \item \textbf{1/2-1/2}: The game ends in a draw.
\end{itemize}

\noindent For more information about notation, refer to the Wikipedia page: \url{https://en.wikipedia.org/wiki/Algebraic_notation_(chess)}.

\subsubsection{Forsyth–Edwards Notation (FEN)}

This is a notation that describes a specific position on a chessboard. It includes 6 fields separated by spaces: the piece placement, whose turn it is to move, castling availability, en passant target square, halfmove clock and fullmove number.
For example, the FEN for the starting position is:
\begin{verbatim}
rnbqkb1r/pppppppp/8/8/8/8/PPPPPPPP/RNBQKBNR w KQkq - 0 1
\end{verbatim}

\vspace{1em}

\noindent Keep in mind this notation is important for the engine to understand the position of the pieces on the board.

\vspace{1em}

\noindent For more information about FEN, refer to the Wikipedia page: \url{https://en.wikipedia.org/wiki/Forsyth%E2%80%93Edwards_Notation}.

\subsubsection{Portable Game Notation (PGN)}

This notation is mostly used for keeping information about the game and a header section with metadata: the name of the event, site, date of play, round, color and name of each player and result. For example, the PGN for a game could look like this:

\begin{lstlisting}[basicstyle=\ttfamily\small, breaklines=true, frame=single, caption={Example of a PGN file}, label={lst:pgn-example}]
[Event "XX Gran Torneo Internacional Aficionado"]
[Site "?"]
[Date "2024.06.25"]
[Round "1"]
[White "Tejedor Barber, Lorenzo"]
[Black "Giron Herranz, Juan"]
[Result "0-1"]
1. e4 c6 2. d4 d5 3. exd5 cxd5 4. Bd3 Nc6 5. c3 Nf6 6. Bf4 Bg4 7. Qb3 Qd7 8. h3 Bh5 9. Nd2 e6 10. Ngf3 a6 11. O-O Be7 12. Rfe1 O-O 13. Re3 b5 14. Ne5 Qb7 15. Nxc6 Qxc6 16. Nf3 Nd7 17. a3 Bg6 18. Bxg6 hxg6 19. Rae1 a5 20. Qd1 b4 21. axb4 axb4 22. h4 bxc3 23. bxc3 Ra3 24. Qd3 Rc8 25. Rc1 Bb4 26. h5 gxh5 27. Ng5 Nf6 28. Be5 Rxc3 29. Rxc3 Qxc3 30. Qe2 Qc1+ 31. Kh2 Bd2 32. Bxf6 Bxe3 33. fxe3 gxf6 34. Nh3 Qb1 35. Nf4 Rc1 36. Nxh5 Rh1+ 37. Kg3
\end{lstlisting}

\noindent For more information about PGN, refer to the Wikipedia page: \url{https://en.wikipedia.org/wiki/Portable_Game_Notation}.

\section{Objectives}

\begin{itemize}
    \item Develop a functional chess engine using alpha-beta pruning as the core search algorithm.
    \item Optimize search efficiency by implementing move ordering, quiescence search, and iterative deepening to improve pruning effectiveness.
    \item Implement transposition tables using Zobrist hashing to store and retrieve previously evaluated board positions efficiently.
    \item Implement multithreading to enable parallel search.
    \item Ensure modularity and efficiency so that the engine can be tested, improved, and integrated into chess-playing applications.
    \item Profile the engine to identify performance bottlenecks and optimize critical sections of the code.
    \item Compare performance metrics against other classical engines to evaluate the impact of implemented optimizations.
\end{itemize}

\section{Work plan}

\begin{enumerate}
    \item Research phase and basic implementation: understand the fundamentals of alpha-beta pruning with minimax and position evaluation. Familiarize with the UCI (Universal Chess Interface) and implement the move generator with its specific exceptions and rules.
    \item Optimization: implement quiescence search and iterative deepening to improve pruning effectiveness.
    \item Optimization: improve search efficiency using transposition tables and Zobrist hashing.
    \item Optimization: implement multithreading to enable parallel search.
    \item Profiling: use a profiler to identify performance bottlenecks and optimize critical sections of the code.
    \item Comparation: use Stockfish to compare efficiency generating tournaments between chess engines.
    \item Analyze the results and write the final report.
\end{enumerate}

%\section{Explicaciones adicionales sobre el uso de esta plantilla}
%Si quieres cambiar el \textbf{estilo del título} de los capítulos del documento, edita el fichero \verb|TeXiS\TeXiS_pream.tex| y comenta la línea \verb|\usepackage[Lenny]{fncychap}| para dejar el estilo básico de \LaTeX.

%Si no te gusta que no haya \textbf{espacios entre párrafos} y quieres dejar un pequeño espacio en blanco, no metas saltos de línea (\verb|\\|) al final de los párrafos. En su lugar, busca el comando  \verb|\setlength{\parskip}{0.2ex}| en \verb|TeXiS\TeXiS_pream.tex| y aumenta el valor de $0.2ex$ a, por ejemplo, $1ex$.

%TFGTeXiS se ha elaborado a partir de la plantilla de TeXiS\footnote{\url{http://gaia.fdi.ucm.es/research/texis/}}, creada por Marco Antonio y Pedro Pablo Gómez Martín para escribir su tesis doctoral. Para explicaciones más extensas y detalladas sobre cómo usar esta plantilla, recomendamos la lectura del documento \texttt{TeXiS-Manual-1.0.pdf} que acompaña a esta plantilla.

%El siguiente texto se genera con el comando \verb|\lipsum[2-20]| que viene a continuación en el fichero .tex. El único propósito es mostrar el aspecto de las páginas usando esta plantilla. Quita este comando y, si quieres, comenta o elimina el paquete \textit{lipsum} al final de \verb|TeXiS\TeXiS_pream.tex|

%\subsection{Texto de prueba}


%\lipsum[2-20]