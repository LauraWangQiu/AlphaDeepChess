\chapter{Introduction}
\label{cap:introduction}

\chapterquote{The most powerful weapon in chess is to have the next move}{David Bronstein}

Chess, one of the oldest and most strategic games in human history, has long been a domain for both intellectual competition and computational research. The pursuit of creating a machine that could compete with the best human players, chess Grandmasters (GM), was present. It was only a matter of time before computation surpassed human computational capabilities.

In 1997, the chess engine Deep Blue made history by defeating the world champion at the time, Garry Kasparov, marking the first time a computer had defeated a reigning world champion in a six-game match under standard chess tournament time controls\footnote{\url{https://en.wikipedia.org/wiki/Deep_Blue_(chess_computer)}}.

Since then, the development of chess engines has advanced rapidly, moving from rule-based systems to AI-driven models. However, classical search algorithms, such as alpha-beta pruning, continue to be fundamental to understanding the basics of efficient search and evaluation of game trees.

%\section{Motivation}

\section{Objectives}

\begin{itemize}
    \item Develop a functional chess engine using alpha-beta pruning as the core search algorithm.
    \item Optimize search efficiency by implementing move ordering, quiescence search, and iterative deepening to improve pruning effectiveness.
    \item Implement transposition tables using Zobrist hashing to store and retrieve previously evaluated board positions efficiently.
    \item Implement multithreading to enable parallel search.
    \item Ensure modularity and efficiency so that the engine can be tested, improved, and integrated into chess-playing applications.
    \item Profile the engine to identify performance bottlenecks and optimize critical sections of the code.
    \item Compare performance metrics against other classical engines to evaluate the impact of implemented optimizations.
\end{itemize}

\section{Work plan}

\begin{enumerate}
    \item Research phase and basic implementation: understand the fundamentals of alpha-beta pruning with minimax and position evaluation. Familiarize with the UCI (Universal Chess Interface) and implement the move generator with its specific exceptions and rules.
    \item Optimization: implement quiescence search and iterative deepening to improve pruning effectiveness.
    \item Optimization: improve search efficiency using transposition tables and Zobrist hashing.
    \item Optimization: implement multithreading to enable parallel search.
    \item Profiling: use a profiler to identify performance bottlenecks and optimize critical sections of the code.
    \item Comparation: use Stockfish to compare efficiency generating tournaments between chess engines.
    \item Analyze the results and write the final report.
\end{enumerate}

%\section{Explicaciones adicionales sobre el uso de esta plantilla}
%Si quieres cambiar el \textbf{estilo del título} de los capítulos del documento, edita el fichero \verb|TeXiS\TeXiS_pream.tex| y comenta la línea \verb|\usepackage[Lenny]{fncychap}| para dejar el estilo básico de \LaTeX.

%Si no te gusta que no haya \textbf{espacios entre párrafos} y quieres dejar un pequeño espacio en blanco, no metas saltos de línea (\verb|\\|) al final de los párrafos. En su lugar, busca el comando  \verb|\setlength{\parskip}{0.2ex}| en \verb|TeXiS\TeXiS_pream.tex| y aumenta el valor de $0.2ex$ a, por ejemplo, $1ex$.

%TFGTeXiS se ha elaborado a partir de la plantilla de TeXiS\footnote{\url{http://gaia.fdi.ucm.es/research/texis/}}, creada por Marco Antonio y Pedro Pablo Gómez Martín para escribir su tesis doctoral. Para explicaciones más extensas y detalladas sobre cómo usar esta plantilla, recomendamos la lectura del documento \texttt{TeXiS-Manual-1.0.pdf} que acompaña a esta plantilla.

%El siguiente texto se genera con el comando \verb|\lipsum[2-20]| que viene a continuación en el fichero .tex. El único propósito es mostrar el aspecto de las páginas usando esta plantilla. Quita este comando y, si quieres, comenta o elimina el paquete \textit{lipsum} al final de \verb|TeXiS\TeXiS_pream.tex|

%\subsection{Texto de prueba}


%\lipsum[2-20]