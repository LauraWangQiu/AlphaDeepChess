\chapter{Work description}
\label{cap:descripcionTrabajo}

%Aquí comienza la descripción del trabajo realizado. Se deben incluir tantos capítulos como sea necesario para describir de la manera más completa posible el trabajo que se ha llevado a cabo. Como muestra la figura \ref{fig:sampleImage}, está todo por hacer.

%\begin{figure}[h]
%	\centering
%	\includegraphics[width = 0.5\textwidth]{Imagenes/Vectorial/Todo.pdf}
%	\caption{Ejemplo de imagen}
%	\label{fig:sampleImage}
%\end{figure}

%Si te sirve de utilidad,  puedes incluir tablas para mostrar resultados, tal como se ve en la tabla \ref{tab:sampleTable}.


%\begin{table}
%	\centering
%	\begin{tabular}{c|c|c}
%		\textbf{Col 1} & \textbf{Col 2} & \textbf{Col 3} \\
%		\hline\hline
%		3 & 3.01 & 3.50\\
%		6 & 2.12 & 4.40\\
%		1 & 3.79 & 5.00\\
%		2 & 4.88 & 5.30\\
%		4 & 3.50 & 2.90\\
%		5 & 7.40 & 4.70\\
%		\hline
%	\end{tabular}
%	\caption{Tabla de ejemplo}
%	\label{tab:sampleTable}
%\end{table}

\section{Where to begin?}

Explain how is organised the different modules of the code and what is UCI in order to help the current reader to recreate a mental map of how to use it.

\section{Basic concepts}

Introduction to the basic concepts of chess.

\subsection{Chess board}

\subsection{Chess pieces}

\subsection{Movement of the pieces}

\subsection{Rules}

White starts first, possible results (win, draw or lose), 30 moves rule, zugzwang?, check, checkmate, ...

\section{Modules}

\subsection{Board}

\subsection{Move generator}

\subsection{Move ordering}

\subsection{Evaluation}

\subsection{Search}

\section{Code implementation}

\subsection{Data representation}

Use of uint64\_t as bitboards to store the board information and other code structures and classes justifying why is efficient.

\subsection{Initialized memory}

Some tables are memory initialized instead of computed, explain it.

\section{Other}

\subsection{Board visualizer using Python}

\subsection{Comparation using Cutechess and Stockfish engine}

\subsection{Profiling}