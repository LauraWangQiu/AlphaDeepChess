\chapter*{Abstract}

\section*{\tituloPortadaEngVal}

Chess engines have played a fundamental role in the advancement of artificial intelligence applied to the game since the mid-20th century. Pioneers such as Alan Turing and Claude Shannon established the theoretical principles that laid the foundation for this field. Building upon these foundations, the evolution of hardware and the refinement of search techniques techniques have enabled significant advancements, such as alpha-beta pruning, an optimization of the minimax algorithm that drastically reduces the number of nodes evaluated in the game tree. Today, Stockfish, the most powerful and open-source chess engine, continues to rely on alpha-beta pruning but also incorporates deep learning techniques and neural networks.

\vspace{1em}

The goal of this project is to develop a chess engine capable of competing against both other engines and human players, using alpha-beta pruning as its core. Additionally, we will analyze the impact of other classical algorithmic techniques such as transposition tables, iterative deepening, and a move generator based on magic bitboards.

\vspace{1em}

The chess engine has finally been uploaded to the Lichess.org platform, where AlphaDeepChess achieved an ELO rating of 1900 while running on a Raspberry Pi 5 equipped with a 2TB transposition table.

\section*{Keywords}

\noindent chess, chess engine, artificial intelligence, alpha-beta pruning, iterative deepening, quiescence search, move ordering, transposition table, zobrist hashing, pext instruction, magic bitboards
