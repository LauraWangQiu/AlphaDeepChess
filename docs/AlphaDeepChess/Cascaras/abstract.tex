\chapter*{Abstract}

\section*{\tituloPortadaEngVal}

Chess engines have played a fundamental role in the advancement of artificial intelligence applied to chess since the mid-20th century. In addition to classical search algorithms, AI is applied to chess through neural networks for position evaluation and reinforcement learning.

\vspace{1em}

\noindent Pioneers such as Alan Turing and Claude Shannon laid the theoretical foundations of the field, such as modeling chess as a search tree, the minimax algorithm, evaluation functions and techniques to reduce the search space. Building on these foundations, the evolution of hardware and the refinement of search techniques have enabled significant advances, such as alpha-beta pruning, an optimization of the minimax algorithm that drastically reduces the number of evaluated nodes in the game tree. Today, \textit{Stockfish}, the most powerful open source chess engine, still relies on alpha-beta pruning, but also incorporates deep learning and neural network techniques.

\vspace{1em}

\noindent The goal of this project is to develop a chess engine capable of competing against both other engines and human players, using minimax with alpha-beta pruning as its core. Additionally, we analyze the impact of other classical algorithmic techniques such as transposition tables, iterative deepening, and a move generator based on magic bitboards.

\vspace{1em}

The chess engine has been uploaded to the Lichess platform, where AlphaDeepChess achieved an ELO rating of 1900 while running on a Raspberry Pi 5 equipped with a 2GB transposition table.

\section*{Keywords}

\noindent chess, chess engine, alpha-beta pruning, iterative deepening, quiescence search, move ordering, transposition table, zobrist hashing, pext instruction, magic bitboards
