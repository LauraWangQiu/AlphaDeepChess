\chapter*{Resumen}

\section*{\tituloPortadaVal}

Los motores de ajedrez han desempeñado un papel fundamental en el avance de la inteligencia artificial aplicada al juego desde mediados del siglo XX. Hoy en día, \textit{Stockfish}, el motor de ajedrez más potente y de código abierto, sigue basándose en la poda alfa-beta, pero también incorpora técnicas de aprendizaje automático.

\vspace{1em}

El objetivo de este proyecto es desarrollar un motor de ajedrez capaz de competir tanto contra otros motores como contra jugadores humanos, utilizando la poda alfa-beta como núcleo del algoritmo. Además, se analiza el impacto de otras técnicas clásicas, como las tablas de transposición, la búsqueda en profundidad iterativa y un generador de movimientos basado en bitboards mágicos.

\vspace{1em}

El motor ha sido subido a la plataforma \textit{Lichess}, donde \textit{AlphaDeepChess} ha alcanzado una puntuación ELO de 1900, ejecutándose en una Raspberry Pi 5 con una tabla de transposiciones de 2GB.

\section*{Palabras clave}
   
\noindent ajedrez, inteligencia artificial, motor de ajedrez, poda alfa-beta, búsqueda en profundidad iterativa, búsqueda quiescente, ordenación de movimientos, tabla de transposiciones, zobrist hashing, bitboards mágicos
