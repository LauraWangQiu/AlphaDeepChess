\chapter*{Resumen}

\section*{\tituloPortadaVal}

Los motores de ajedrez han desempeñado un papel fundamental en el avance de la inteligencia artificial aplicada al ajedrez desde mediados del siglo XX. Además de los algoritmos clásicos de búsqueda, la IA se aplica al ajedrez mediante redes neuronales para evaluar posiciones y aprendizaje por refuerzo.

\vspace{1em}

\noindent Pioneros como Alan Turing y Claude Shannon sentaron las bases teóricas del campo, como el modelado del ajedrez como un árbol de búsqueda, el algoritmo minimax, las funciones de evaluación y técnicas para reducir el espacio de búsqueda. Sobre estos cimientos, la evolución del hardware y el perfeccionamiento de las técnicas de búsqueda han permitido avances significativos, como la poda alfa-beta, una optimización del algoritmo minimax que reduce drásticamente el número de nodos evaluados en el árbol del juego. Hoy en día, \textit{Stockfish}, el motor de ajedrez más potente y de código abierto, sigue basándose en la poda alfa-beta, pero también incorpora técnicas de aprendizaje profundo y redes neuronales.

\vspace{1em}

El objetivo de este proyecto es desarrollar un motor de ajedrez capaz de competir tanto contra otros motores como contra jugadores humanos, utilizando la poda alfa-beta como núcleo del algoritmo. Además, se analiza el impacto de otras técnicas clásicas, como las tablas de transposición, la búsqueda en profundidad iterativa y un generador de movimientos basado en bitboards mágicos.

\vspace{1em}

El motor ha sido subido a la plataforma Lichess, donde AlphaDeepChess ha alcanzado una puntuación ELO de 1900, ejecutándose en una Raspberry Pi 5 con una tabla de transposiciones de 2GB.

\section*{Palabras clave}
   
\noindent ajedrez, motor de ajedrez, poda alfa-beta, búsqueda en profundidad iterativa, búsqueda quiescente, ordenación de movimientos, tabla de transposiciones, zobrist hashing, instrucción pext, bitboards mágicos 
