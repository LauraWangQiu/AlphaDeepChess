\chapter*{Resumen}

\section*{\tituloPortadaVal}

Los motores de ajedrez han desempeñado un papel fundamental en el avance de la inteligencia artificial aplicada al ajedrez desde mediados del siglo XX. Pioneros como Alan Turing y Claude Shannon establecieron los principios teóricos que sentaron las bases de este campo. Sobre estos cimientos, la evolución del hardware y el perfeccionamiento de técnicas de búsqueda permitieron importantes avances, como la poda alfa-beta, una optimización del algoritmo minimax que reduce drásticamente el número de nodos evaluados en el árbol de juego. Hoy en día, Stockfish, el motor de ajedrez más potente y de código abierto, sigue basándose en técnicas algorítmicas clásicas, pero también incorpora deep learning y redes neuronales.

\vspace{1em}

El objetivo de este proyecto es desarrollar un motor de ajedrez capaz de competir tanto contra otros motores como contra jugadores humanos, utilizando la poda alfa-beta como núcleo del algoritmo. Además, se analizará el impacto de otras técnicas algorítmicas clásicas, como las tablas de transposición, la búsqueda en profundidad iterativa y un generador de movimientos basado en bitboards mágicos.

\vspace{1em}

Finalmente, el motor de ajedrez ha sido subido a la plataforma de Lichess, donde AlphaDeepChess ha alcanzado una puntuación ELO de 1900, ejecutándose en una Raspberry Pi 5 con una tabla de transposiciones de 2TB.

\section*{Palabras clave}
   
\noindent ajedrez, motor de ajedrez, poda alfa-beta, búsqueda en profundidad iterativa, búsqueda quiescente, ordenación de movimientos, tabla de transposiciones, zobrist hashing, instrucción pext, bitboards mágicos 
