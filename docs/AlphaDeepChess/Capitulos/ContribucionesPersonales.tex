\chapter{Personal Contributions}\label{cap:contribucionesPersonales}

The following section details the individual contributions made by each team member, Juan and Yi, throughout the development of the AlphaDeepChess project. Each contribution is listed to provide transparency regarding the division of work, highlight specific areas of responsibility, and acknowledge the expertise and effort invested by each author.

\section*{Juan Girón Herranz}

\begin{itemize}[itemsep=1pt]

    \item Contributed to the alpha-beta pruning algorithm. Taking part in the research and implementation of the iterative deepening and the algorithm core.

    \item Designed and implemented the bitboard-based move generator, then optimized the calculation of the slider pieces moves with magic bitboards technique and the PEXT hardware instruction. Additionally, integrated of the move generator in the search algorithm.

    \item Involved in the implementation of the board data structure, with emphasis on the game state bit field design.

    \item Design and Developed the move data structure, which was also optimized as a bit field to reduce space consumption.    

    \item Designed and implemented the auxiliary data structures for rows, columns, diagonals, and directions. These structures played a key role in simplifying and optimizing bitboard masking operations, enabling more efficient move generation and attack pattern calculations.
    
    \item Researched, designed and implemented the transposition table using Zobrist hashing, with total integration in the alpha-beta search.

    \item Introduced MVV-LVA (Most Valuable Victim, Least Valuable Aggressor), then research and enhanced the algorithm with killer-move heuristics.

    \item Researched and developed the quiescence search enhancement to avoid the horizon effect in the alpha-beta pruning.

    \item Developed and tested the search with the late move reductions technique.

    \item Developed the tampering evaluation, adjusting the weight for the middlegame and endgame evaluations, also contributed to optimizations in the implementation for king safety and piece mobility.

    \item Created the algorithm to detect threefold repetition using the game's position history and its implementation in the search.

    \item Implemented part of the UCI command parsing and communication for engine integration.

    \item Conducted multiple 100-game matches using CuteChess between engine versions to measure the impact of each optimization.

    \item Created of hundreds of unit tests, which have been a fundamental part of finding bugs and ensuring code quality. This includes Perft testing of the move generator, a standard technique that counts all possible legal positions up to a certain depth to ensure the correctness of move generation in the chess engine.

    \item Contributed to the development of a helper GUI in Python to facilitate interactive testing of the engine, with support for the UCI protocol.    

    \item Used Linux's \texttt{perf} tool, analyzed the CPU overhead of the different parts of the chess engine.

    \item Compiled and deployed the engine on a Raspberry Pi 5, configuring it as a Lichess.org bot.  Running under limited hardware resources, the engine achieved competitive ELO ratings while demonstrating our code's efficiency and portability.

\end{itemize}

\section*{Yi Wang Qiu}

\begin{itemize}[itemsep=1pt]
    \item Responsible for the architectural design and full implementation of the alpha-beta pruning algorithm, established as the foundational search technique of the engine. This algorithm was enhanced through iterative refinement, such as aspiration windows, and theoretical benchmarking, enabling effective traversal of the game tree while significantly reducing the computational overhead associated with brute-force minimax strategies.

    \item Developed an optimized multithreaded search version incorporating the Young Brothers Wait Concept (YBWC), a parallelization paradigm specifically tailored for game tree evaluation.

    \item Engineered the parsing and command interpretation system compliant with the Universal Chess Interface (UCI) protocol. This subsystem ensures seamless bidirectional communication between the chess engine and external graphical user interfaces, testing suites, and benchmarking frameworks.

    \item Designed and implemented the core engine abstractions, \texttt{Square} and \texttt{Board} classes, which supports the representation and manipulation of chess positions. These classes encapsulate critical logic such as coordinate translation or position translation from FEN, piece tracking, castling rights, and \textit{en passant} possibilities, all integrated with a bitboard backend. This design allows high-level readability while preserving low-level computational performance.

    \item Constructed the internal board representation model using 64-bit bitboards. This representation supports highly efficient binary operations such as masking, shifting, and logical conjunctions to simulate piece movement and board updates.

    \item Developed a modular and extensible evaluation system capable of quantifying chess positions through multiple heuristic lenses. The implemented strategies range from basic material balance (expressed in centipawns) to more sophisticated models that incorporate positional features such as game phase, piece activity, mobility scoring, and king vulnerability or safety. These heuristics were designed to be dynamically weighted depending on the stage of the game (opening, middlegame, or endgame).

    \item Integrated and calibrated the precomputed positional data structures, including piece-square tables to accelerate the static evaluation of positions.

    \item Designed and authored a suite of automated Python scripts for orchestrating engine versus engine tournaments and performance benchmarking using \textit{Cutechess CLI}. These scripts included configurable match parameters like search time, depth of search, number of games, book of openings or initial positions. They were essential in enabling the reproducibility of experiments, comparison of successive versions of the engine, and quantification of the impact of algorithmic refinements.

    \item Established a robust continuous integration and delivery (CI/CD) pipeline using GitHub Actions. This infrastructure automated the build, deployment, and testing stages of the engine. Used the above Python scripts to automate the tournaments in an independent machine to avoid wasting time and computation capacity while still developing.

    \item Contributed to the frontend layer of the project by prototyping a graphical user interface (GUI) in Python, designed to allow interactive execution of the engine in a visual environment. The GUI included subprocess communication features, move display, and optional positional evaluations. Although later iterations focused on headless execution, this interface was key during early debugging and demonstration phases.

    \item Authored detailed and structured online documentation describing the engine's internal architecture, modular hierarchy, function-level responsibilities, and usage guidelines. The documentation was designed not only as an educational resource for future contributors, but also as a formal exposition of the system's logic for academic evaluation purposes like this exact document. It includes illustrative diagrams or graphs, code references, and configuration examples to support transparency and reproducibility.
\end{itemize}
