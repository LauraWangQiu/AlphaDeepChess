\chapter{Conclusions and Future Work}
\label{cap:conclusiones}

AlphaDeepChess has proven to be an effective implementation of classical techniques focused on alpha-beta pruning. Despite the rise of neural network-based engines, this project demonstrates that well-optimized traditional approaches remain competitive.

\vspace{1em}

\noindent The engine achieved an ELO score of 1900 on Lichess, running on a Raspberry Pi 5 with a 2TB transposition table, confirming its efficiency even on limited hardware.
Link to the lichess engine profile: \url{https://lichess.org/@/AlphaDeepChess}

\vspace{1em}

\noindent Incorporating techniques such as transposition tables, iterative deepening, move ordering heuristics, and magic bitboards contributed significantly to the engine's performance. The use of Cutechess testing and comparisons with Stockfish allowed this impact to be measured.

\vspace{1em}

\noindent The next steps to be implemented would be the application of neural networks (NNUE) which, although intended for CPUs, could be thought of as a streamlined evaluation with GPUs as performed by Leela Chess Zero.
