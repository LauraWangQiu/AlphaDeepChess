\chapter{Conclusions and Future Work}

We have successfully developed a competitive chess engine based on an enhanced minimax search with alpha-beta pruning, a straightforward materialistic evaluation, and a Most Valuable Victim-Least Valuable Aggressor (MVV-LVA) move-ordering heuristic.

\vspace{1em}

\noindent Building on this basic implementation, we have researched and analyzed the following algorithmic techniques to further improve the bot's playing strength:

\begin{itemize}[itemsep=1pt]
    \item Transposition tables with Zobrist hashing: provided an increase in performance by avoiding redundant position evaluations.
    \item Move generator with magic bitboards and PEXT instructions: substantial improvement in move generation.
    
    \item Evaluation with king safety and piece mobility parameters: showed no clear improvement, likely due to the added computational cost and the difficulty of correlating these factors directly with positional strength.
    \item Multithreaded search: requires further optimization of data structures and algorithms for concurrent access to yield tangible gains.
    \item Search with Late Move Reductions: did not improve performance, as our current move-ordering heuristic is not strong enough to support such aggressive pruning.
\end{itemize}

\noindent The engine achieved an ELO rating of 1900 on Lichess while running on a Raspberry Pi 5 with a 2GB transposition table, demonstrating its efficiency even on resource-constrained hardware.

\vspace{1em}

\noindent The engine's Lichess profile can be found at:\\
\url{https://lichess.org/@/AlphaDeepChess}

\vspace{1em}

\section{Future Work}

\noindent Potential avenues for further improvement include:

\begin{itemize}[itemsep=1pt]
    \item Neural-network evaluation and move ordering:  
    Application of neural networks for the evaluation function and the move ordering heuristic, following the steps of \textit{Stockfish}, the actual best chess engine. This could unlock more aggressive pruning strategies.
    \item Advanced pruning techniques:  
    With a stronger evaluation and ordering heuristic, revisit and tune the actual Late Move Reductions, and implement null move pruning technique.
    \item Multithreading optimizations:  
    Refine concurrent search algorithm and redesign shared data structures like transposition tables to support concurrent access.
\end{itemize}
