\chapter{Conclusiones y trabajo futuro}

Hemos desarrollado con éxito un motor de ajedrez competitivo basado en una búsqueda minimax mejorada con poda alfa-beta, una evaluación materialista directa y una heurística de ordenación de jugadas MVV-LVA (Most Valuable Victim-Least Valuable Aggressor).

\vspace{1em}

\noindent Partiendo de esta implementación básica, hemos investigado y analizado las siguientes técnicas algorítmicas para mejorar aún más la fuerza de juego del bot:

\begin{itemize}[itemsep=1pt]
    \item \textit{Tablas de transposición con Zobrist hashing}: proporcionó un aumento en el rendimiento al evitar evaluaciones de posición redundantes.
    \item \textit{Generador de jugadas con bitboards mágicos e instrucciones PEXT}: mejora sustancial en la generación de jugadas.
    \item \textit{Evaluación con parámetros de seguridad de rey y movilidad de pieza}: no mostró una mejora clara, probablemente debido al coste computacional añadido y a la dificultad de correlacionar estos factores directamente con la fuerza posicional.
    \item \textit{Búsqueda multihilo}: requiere una mayor optimización de las estructuras de datos y algoritmos de acceso concurrente para producir ganancias tangibles.
    \item \textit{Búsqueda con reducciones de movimientos tardíos}: no mejoró el rendimiento, ya que nuestra heurística actual de ordenación de movimientos no es lo suficientemente fuerte como para soportar una poda tan agresiva.
\end{itemize}

\noindent El motor alcanzó una puntuación Elo de 1900 en \textit{Lichess} mientras se ejecutaba en una Raspberry Pi 5 con una tabla de transposición de 2GB, lo que demuestra su eficiencia incluso en hardware con recursos limitados.

\vspace{1em}

\noindent El perfil del motor de ajedrez puede encontrarse en: \\
\url{https://lichess.org/@/AlphaDeepChess}

\newpage
\section{Trabajo futuro}

\noindent Entre las posibles vías de mejora figuran las siguientes:

\begin{itemize}[itemsep=1pt]
    \item \textit{Evaluación por redes neuronales y ordenación de jugadas}:
    Aplicación de redes neuronales para la función de evaluación y la heurística de ordenación de jugadas, siguiendo los pasos de \textit{Stockfish}, el mejor motor de ajedrez actual. Esto podría desbloquear estrategias de poda más agresivas.
    \item \textit{Técnicas avanzadas de poda}:  
    Con una evaluación más sólida y una heurística de ordenación, revisar y ajustar las reducciones de movimientos tardíos e implementar la técnica de poda de movimientos nulos.
    \item \textit{Optimizaciones multihilo}:  
    Perfeccionar el algoritmo de búsqueda concurrente y rediseñar las estructuras de datos compartidos como las tablas de transposición para soportar el acceso concurrente.
\end{itemize}
