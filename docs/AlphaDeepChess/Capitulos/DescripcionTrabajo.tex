\chapter{Work description}
\label{cap:descripcionTrabajo}

We designed a simple structure to organize all of the scripts. As we are implementing different versions of the engine, there are specific modules that are compulsory to have well distinguished and possibly selected between them. Those modules are presented in the following Section~\ref{sec:modules}.

\vspace{1em}

\noindent The key point is how to represent each required structure, including the board, chess pieces, precomputed tables, transposition tables, etc., as discussed in Section~\ref{sec:code}.

\section{Modules}
\label{sec:modules}

\subsection{Board}

\subsection{Move generator}

\subsection{Move ordering}

\subsection{Evaluation}

\subsection{Search}

\section{Code implementation}
\label{sec:code}

\subsection{Data representation}

Use of uint64\_t as bitboards to store the board information and other code structures and classes justifying why is efficient.

\begin{lstlisting}
// Example of code in LaTeX
#include <iostream>

int main() {
    std::cout << "Hello world! << std::endl;
}
\end{lstlisting}

\subsection{Initialized memory}

Some tables are memory initialized instead of computed, explain it.

\section{Additional tools and work}

\subsection{Board visualizer using Python}

\subsection{Comparation using Cutechess and Stockfish engine}
