\chapter{Introduction}\label{cap:introduction}
\renewcommand{\figurename}{Figure}

Chess, one of the oldest strategy games in human history, has long been a domain for both intellectual competition and computational research. The pursuit of creating a machine that could compete with the best human players, chess Grandmasters, was present. It was only a matter of time before computation surpassed human capabilities.

\vspace{1em}

\noindent Today, we find ourselves in an era where chess engines have reached unprecedented strength. \textit{Stockfish}, the strongest chess engine in the world, continues to rely on an improved minimax algorithm with alpha-beta pruning, additionally incorporating machine learning techniques. Inspired by it, we started this project: \textit{AlphaDeepChess}.

\vspace{1em}

\noindent We present \textit{AlphaDeepChess}, a chess engine based on minimax with alpha-beta pruning that relies solely on classical algorithms and implements optimization techniques based on current knowledge about chess engines. Notably, the engine has achieved an Elo rating of around 1900 on \textit{Lichess}, demonstrating its competitive strength among online chess engines.

\vspace{1em}

\noindent The source code of our engine is available at:\\
\url{https://github.com/LauraWangQiu/AlphaDeepChess}

\newpage

\section{Objectives}\label{sec:objectives}

\noindent The objectives of this project are the following:

\begin{itemize}[itemsep=1pt]
    \item Develop a chess engine based on minimax search with alpha-beta pruning that follows the UCI protocol~\cite{UciProtocol}. The engine will be a console application capable of playing chess against humans or other engines, analyzing and evaluating positions to determine the best legal move.
    \item Implement various known optimization techniques, including move ordering, quiescence search, iterative deepening, transposition tables, multithreading, and a move generator based on magic bitboards.
    \item Measure the impact of these optimization techniques and profile the engine to identify performance bottlenecks.
    \item Upload the engine to \texttt{lichess.org} and compete against other chess engines.
\end{itemize}

\section{Work plan}

The project was divided into several phases, each focusing on a specific aspect of the engine's development. The timeline for each phase is as follows:

\begin{enumerate}
    \item \textit{Research phase and basic implementation}: understand the fundamentals of minimax with alpha-beta pruning and position evaluation. Familiarize with the UCI (Universal Chess Interface) and implement the move generator with its specific exceptions and rules.
    \item \textit{Optimization}: implement quiescence search and iterative deepening to improve pruning effectiveness.
    \item \textit{Optimization}: improve search efficiency using transposition tables and Zobrist hashing.
    \item \textit{Optimization}: implement multithreading to enable parallel search.
    \item \textit{Profiling}: use a profiler to identify performance bottlenecks and optimize critical sections of the code.
    \item \textit{Benchmarking}: use \textit{Stockfish} to compare efficiency generating tournaments between chess engines and estimate the performance of the engine. Also, compare different versions of the engine to evaluate the impact of optimizations.
    \item Analyze the results and write the final report.
\end{enumerate}

\noindent In the following~\cref{sec:basicConcepts}, we will explain about the basic concepts of chess, but if you already have the knowledge, we recommend to advance directly to the next~\cref{cap:estadoDeLaCuestion}.

\section{Chess fundamentals}\label{sec:basicConcepts}

\noindent Chess is a board game where two players, taking white pieces and black pieces respectively, compete to be the first to checkmate the opponent. In chess, to \textit{capture} means to move one of your pieces to a square occupied by an opponent's piece, thereby removing it from the board. Checkmate occurs when the king is under threat of capture (known as check) by a piece or pieces of the enemy, and there is no legal way to escape or remove the threat.

\vspace{1em}

\noindent A chess engine consists of a software program that analyzes chess positions and returns optimal moves depending on its configuration. In order to help users to use these engines, the chess community agreed on creating an open communication protocol called \textbf{Universal Chess Interface} (commonly referred to as UCI), that allows interacting with chess engines through user interfaces.

\vspace{1em}

\noindent A chess game takes place on a chessboard with specific rules governing the movements and interactions of the pieces. This section introduces the fundamental concepts necessary to understand how chess is played.

\subsection*{Chessboard}\label{sec:chessboard}

A chessboard is a game board of 64 squares arranged in 8 rows and 8 columns, as shown in the following figure:

\begin{figure}[H]
    \centering
    \newchessgame
    \chessboard[setpieces={},showmover=false]
    \caption*{Empty chessboard.}\label{fig:chessboard}
\end{figure}

\vspace{1em}

\noindent To refer to each of the squares we mostly use \textbf{algebraic notation} using the numbers from 1 to 8 and the letters from ``a'' to ``h''. There are also other notations like descriptive notation (now obsolete) or ICCF numeric notation, which solves the problem of chess pieces having different abbreviations depending on the language. For example, $g5$ refers to the following square:

\begin{center}
    \newchessgame % Crear un nuevo tablero vacío
    \chessboard[
        setpieces={}, % No piezas en el tablero
        showmover=false,
        markstyle=circle, color=red, markfield=g5, % Marcar g5 con un círculo rojo
        pgfstyle=straightmove, color=blue, % Estilo de flechas rectas y azules
        markmoves={g1-g5, a5-g5}, % Flechas desde g1 → g5 y a5 → g5
        arrow=to % Flechas con punta
    ]
\end{center}

\noindent It is important to know that when placing a chessboard in the correct orientation, there should always be a white square in the bottom-right corner or a black square in the bottom-left corner.

\subsection*{Chess pieces}

There are 6 types of chess pieces: king, queen, rook, bishop, knight, and pawn, and each side has 16 pieces, as shown in~\cref{tab:number-of-pieces}.

\vspace{1em}

\begin{table}[b]
    \centering
    \resizebox{\textwidth}{!}{
    \begin{tabular}{|c|c|c|c|}
        \hline
        Piece & White Pieces & Black Pieces & Number of Pieces \\ \hline
        King           & \WhiteKingOnWhite     & \BlackKingOnWhite     & 1                         \\ \hline
        Queen          & \WhiteQueenOnWhite    & \BlackQueenOnWhite    & 1                         \\ \hline
        Rook           & \WhiteRookOnWhite\WhiteRookOnWhite & \BlackRookOnWhite\BlackRookOnWhite & 2 \\ \hline
        Bishop         & \WhiteBishopOnWhite\WhiteBishopOnWhite & \BlackBishopOnWhite\BlackBishopOnWhite & 2 \\ \hline
        Knight         & \WhiteKnightOnWhite\WhiteKnightOnWhite & \BlackKnightOnWhite\BlackKnightOnWhite & 2 \\ \hline
        Pawn           & \WhitePawnOnWhite\WhitePawnOnWhite\WhitePawnOnWhite\WhitePawnOnWhite\WhitePawnOnWhite\WhitePawnOnWhite\WhitePawnOnWhite\WhitePawnOnWhite & \BlackPawnOnWhite\BlackPawnOnWhite\BlackPawnOnWhite\BlackPawnOnWhite\BlackPawnOnWhite\BlackPawnOnWhite\BlackPawnOnWhite\BlackPawnOnWhite & 8 \\ \hline
    \end{tabular}
    }
    \caption{Number of chess pieces by type and color.}\label{tab:number-of-pieces}
\end{table}

\noindent The starting position of the chess pieces on a chessboard is described in~\cref{fig:start-pos}.

\begin{figure}
    \begin{minipage}{0.45\textwidth}
        \centering
        \newchessgame
        \chessboard[showmover=true]
    \end{minipage}
    \hfill
    \begin{minipage}{0.45\textwidth}
        \centering
        \newchessgame
        \chessboard[
            showmover=false,
            setpieces={},
            pgfstyle=color, opacity=0.2,
            color=red,
            markfields={a1,a2,a3,a4,a5,a6,a7,a8,b1,b2,b3,b4,b5,b6,b7,b8,c1,c2,c3,c4,c5,c6,c7,c8,d1,d2,d3,d4,d5,d6,d7,d8},
            color=blue!50,
            markfields={e1,e2,e3,e4,e5,e6,e7,e8,f1,f2,f3,f4,f5,f6,f7,f8,g1,g2,g3,g4,g5,g6,g7,g8,h1,h2,h3,h4,h5,h6,h7,h8}
        ]
    \end{minipage}
    \caption{Starting position.}\label{fig:start-pos}
\end{figure}

\noindent The smaller white square next to the board indicates which side is to move in the current position. If the square is white, it means it is white's turn to move; if the square is black, it means it is black's turn to move. Notice that the queen and king are placed in the center columns. The queen is placed on a square of its color, while the king is placed on the remaining central column. The rest of the pieces are positioned symmetrically, as shown in~\cref{fig:start-pos}. This means that the chessboard is divided into two sides relative to the positions of the king and queen at the start of the game. Red is queen's side and blue is king's side.

\subsection*{Movement of the pieces}

In the following sections, we describe the movement rules for each type of piece, including their unique abilities and special moves. These are fundamental to understand playing and analyzing the game.

\subsubsection*{Pawn}

The pawn can move one square forward, but it can only capture pieces one square diagonally, as shown in the top right chessboard in the figure below. On its first move, the pawn has the option to move two squares forward. If a pawn reaches the last row of the opponent's side, it promotes to any other piece (except for a king). Promotion is a term to indicate the mandatory replacement of a pawn with another piece, usually providing a significant advantage to the player who promotes. This is also illustrated in the two lower boards of the figure below.

\newpage

\begin{figure}[H]
    \centering
    \begin{minipage}{0.22\textwidth}
        \centering
        \setchessboard{boardfontsize=12pt,labelfontsize=8pt}
        \newchessgame
        \chessboard[
            setpieces={Pe2, Pc4, pc5, Pg3, pf6},
            showmover=false,
            pgfstyle=straightmove, color=blue,
            markmoves={e2-e3,e2-e4,g3-g4,f6-f5},
            arrow=to
        ]
    \end{minipage}
    \hspace{0.04\textwidth}
    \begin{minipage}{0.22\textwidth}
        \centering
        \setchessboard{boardfontsize=12pt,labelfontsize=8pt}
        \newchessgame
        \chessboard[
           setpieces={Pe2, Pc4, pc5, Pg3, pf6},
           showmover=false,
           pgfstyle=straightmove, color=red,
           markmoves={e2-d3,e2-f3,c4-b5,c4-d5,g3-f4,g3-h4,f6-e5,f6-g5,c5-b4,c5-d4},
           arrow=to
       ]
    \end{minipage}
    \\[1em]
    \begin{minipage}{0.22\textwidth}
        \centering
        \setchessboard{boardfontsize=12pt,labelfontsize=8pt}
        \newchessgame
        \chessboard[
            setpieces={Pe7},
            showmover=false,
            pgfstyle=straightmove, color=blue,
            markmoves={e7-e8},
            arrow=to
        ]
    \end{minipage}
    \hspace{0.04\textwidth}
    \begin{minipage}{0.22\textwidth}
        \centering
        \setchessboard{boardfontsize=12pt,labelfontsize=8pt}
        \newchessgame
        \chessboard[
            setpieces={Qe8},
            showmover=false
        ]
    \end{minipage}
    \caption*{Pawn's movement, attack, and promotion.}
\end{figure}

\noindent There is a specific capture movement which is \textit{en passant}. This move allows a pawn that has moved two squares forward from its starting position to be captured by an opponent's pawn as if it had only moved one square, as shown in the figure below. The capturing pawn must be on an adjacent file and can only capture the \textit{en passant} pawn immediately after it moves.

\begin{figure}[H]
    \centering
    \begin{minipage}{0.3\textwidth}
        \centering
        \setchessboard{boardfontsize=15pt,labelfontsize=10pt}
        \newchessgame[black]
        \chessboard[
            setpieces={ke8,Ke1,pd7,Pe5},
            showmover=false,
            pgfstyle=straightmove, color=blue,
            markmoves={d7-d5},
            arrow=to
        ]
    \end{minipage}
    \hfill
    \begin{minipage}{0.3\textwidth}
        \centering
        \setchessboard{boardfontsize=15pt,labelfontsize=10pt}
        \newchessgame
        \chessboard[
            setpieces={ke8,Ke1,pd5,Pe5},
            showmover=false,
            pgfstyle=straightmove, color=red,
            markmoves={e5-d6},
            arrow=to
        ]
    \end{minipage}
    \hfill
    \begin{minipage}{0.3\textwidth}
        \centering
        \setchessboard{boardfontsize=15pt,labelfontsize=10pt}
        \newchessgame
        \chessboard[
            setpieces={ke8,Ke1,Pd6},
            showmover=false
        ]
    \end{minipage}
    \caption*{\textit{En passant}.}
\end{figure}

\subsubsection*{Rook}

The rook can move any number of squares horizontally or vertically. It can also capture pieces in the same way, as shown in the following figure:

\begin{figure}[H]
    \centering
    \newchessgame
    \chessboard[
        setpieces={Rd4,Rg6,Ng2,bc6},
        showmover=false,
        pgfstyle=straightmove, color=blue,
        markmoves={d4-a4,d4-h4,d4-d1,d4-d8,g6-h6,g6-g8,g6-c6,g6-g3},
        arrow=to
    ]
    \caption*{Rook's movement.}
\end{figure}

\subsubsection*{Knight}

The knight moves in an L-shape: two squares in one direction and then one square perpendicular to that direction. The knight can jump over other pieces, making it a unique piece in terms of movement. It can also capture pieces in the same way. This is shown in the following figure:

\begin{figure}[H]
    \centering
    \newchessgame
    \chessboard[
        setpieces={Nf3,Na8,pb7},
        showmover=false,
        pgfstyle=straightmove, color=blue,
        markmoves={f3-e5,f3-e1,f3-g5,f3-g1,f3-d2,f3-d4,f3-h2,f3-h4,a8-b6,a8-c7},
        arrow=to
    ]
    \caption*{Knight's movement.}
\end{figure}

\subsubsection*{Bishop}

The bishop can move any number of squares diagonally. It can also capture pieces in the same way. Considering that each side has two bishops, one bishop moves on light squares and the other on dark squares. This is shown in the following figure:

\begin{figure}[H]
    \centering
    \newchessgame
    \chessboard[
        setpieces={Bc4,Bf6,nb2,Qe7},
        showmover=false,
        pgfstyle=straightmove, color=blue,
        markmoves={c4-a2,c4-g8,c4-f1,c4-a6,f6-h4,f6-h8,f6-b2},
        arrow=to
    ]
    \caption*{Bishop's movement.}
\end{figure}

\subsubsection*{King}

The king is a crucial piece in chess, as the game ends when one player checkmates the opponent's king. The king can move one square in any direction: horizontally, vertically, or diagonally, and can also capture pieces in the same way. 

\begin{figure}[H]
    \centering
    \begin{minipage}[t]{0.45\textwidth}
       \centering
       \newchessgame
       \chessboard[
           setpieces={Ke4},
           showmover=false,
           pgfstyle=straightmove, color=blue,
           markmoves={e4-e5,e4-e3,e4-d4,e4-f4,e4-d5,e4-f5,e4-d3,e4-f3},
           arrow=to
       ]
   \end{minipage}
   \hspace{0.1em}
   \begin{minipage}[t]{0.45\textwidth}
       \centering
       \newchessgame
       \fenboard{8/8/1p6/p1pBk1p1/P1P2p1p/1P1K1P1P/1b4P1/8 w - - 0 1}
       \chessboard[
           markstyle=circle, color=red, markfields={d4,c3,e4,e3},
           pgfstyle=straightmove, color=blue,
           markmoves={d3-c2,d3-d2,d3-e2},
           arrow=to
       ]
   \end{minipage}
   \caption*{King's movement.}
\end{figure}


\noindent As shown in the figure above, the white king cannot move to a square that is under attack by an opponent's piece:

\begin{itemize}[itemsep=1pt]
    \item $c3$ is attacked by the black bishop on $b2$.
    \item $d4$ is attacked by the black bishop on $b2$, pawn on $c5$ and black king on $e5$.
    \item $e3$ is attacked by the black pawn on $f4$.
    \item $e4$ is attacked by the black king on $e5$.
\end{itemize}

\paragraph{Castling} is a special move which involves moving the king two squares towards a rook and moving the rook to the square next to the king. Castling has specific conditions which are:

\begin{itemize}[itemsep=1pt]
    \item Neither the king nor the rook involved in castling must have moved previously.
    \item There must be no pieces between the king and the rook.
    \item The king cannot be in check, move through a square under attack, or end up in check.
\end{itemize}

\begin{figure}[H]
    \centering
    \begin{minipage}{0.3\textwidth}
        \centering
        \setchessboard{boardfontsize=15pt,labelfontsize=10pt}
        \newchessgame
        \chessboard[
            setpieces={Ke1,Ra1,Rh1,ke8,rb8,rh8,bf8},
            showmover=false,
            pgfstyle=straightmove, color=blue,
            markmoves={e1-c1, e1-g1},
            arrow=to
        ]
    \end{minipage}
    \hspace{0.1em}
    \begin{minipage}{0.3\textwidth}
        \centering
        \setchessboard{boardfontsize=15pt,labelfontsize=10pt}
        \newchessgame
        \chessboard[
            setpieces={Kg1,Ra1,Rf1,ke8,rb8,rh8,bf8},
            showmover=false
        ]
    \end{minipage}
    \hspace{0.1em}
    \begin{minipage}{0.3\textwidth}
        \centering
        \setchessboard{boardfontsize=15pt,labelfontsize=10pt}
        \newchessgame
        \chessboard[
            setpieces={Kc1,Rd1,Rh1,ke8,rb8,rh8,bf8},
            showmover=false
        ]
    \end{minipage}
    \caption*{Castling situation.}
\end{figure}

\noindent In the figure above, the white king can castle on either the king's side (short castling) or the queen's side (long castling) as long as the rooks have not been moved from their starting position, but the black king cannot castle because there is a bishop on $f8$ interfering with the movement and the rook on the queen's side has been moved to $b8$.

\newpage

\subsubsection*{Queen}

The queen can move any number of squares in any direction: horizontally, vertically, or diagonally. It can also capture pieces in the same way. This is shown in the following figure:

\begin{figure}[H]
    \centering
    \newchessgame
    \chessboard[
        setpieces={Qd4},
        showmover=false,
        pgfstyle=straightmove, color=blue,
        markmoves={d4-a4,d4-h4,d4-d1,d4-d8,d4-a1,d4-h1,d4-h8,d4-a8},
        arrow=to
    ]
    \caption*{Queen's movement.}\label{fig:queen-movement}
\end{figure}

\subsection*{Rules}\label{sec:rules}

The rules of chess follow the official regulations established by FIDE~\cite{LawsOfChess}. As we have already said, the objective of each player is to checkmate the opponent's king, meaning the king is under attack and cannot escape.

\vspace{1em}

\noindent In every game, white starts first, and the possible results of each game can be win for white, win for black or draw. A draw or tie could be caused by different conditions:

\begin{enumerate}
    \item \textit{Stalemate}: the player whose turn it is to move has no legal moves, and their king is not in check.
    \item Insufficient material: neither player has enough pieces to checkmate. Those cases are king vs king, king and bishop vs king, king and knight vs king, and king and bishop vs king and bishop with the bishops on the same color.
    \item \textit{Threefold repetition}: it occurs when same position happens three times during the game, with the same player to move and the same possible moves (including castling and \textit{en passant}).
    \item \textit{Fifty-move rule}: if 50 consecutive moves are made by both players without a pawn move or a capture, the game can be declared a draw.
    \item \textit{Mutual agreement}: both players can agree to a draw at any point during the game.
    \item \textit{Dead position}: a position in which neither player can achieve checkmate by any series of legal moves, making further progress impossible. This includes the case of insufficient material, where it is not possible to checkmate the opponent regardless of the moves played. In these situations, the game is immediately declared a draw because the main objective to checkmate cannot be accomplished.
\end{enumerate}

\begin{figure}[H]
    \centering
    \begin{minipage}[t]{0.3\textwidth}
        \centering
        \setchessboard{boardfontsize=13pt,labelfontsize=10pt}
        \newchessgame
        \chessboard[
            setfen={2k5/8/8/3QB3/8/4K3/8/8 b - - 0 1},
            markstyle=circle, color=red, markfields={b8,b7,c7,d7,d8},
            pgfstyle=straightmove, color=blue,
            markmoves={d5-d8,d5-a8,e5-b8},
            arrow=to
        ]
    \end{minipage}
    \hfill
    \begin{minipage}[t]{0.3\textwidth}
        \centering
        \setchessboard{boardfontsize=13pt,labelfontsize=10pt}
        \newchessgame
        \chessboard[
            setfen={8/8/8/4k3/4B3/4K3/8/8 w - - 0 1}
        ]
    \end{minipage}
    \hfill
    \begin{minipage}[t]{0.3\textwidth}
        \centering
        \setchessboard{boardfontsize=13pt,labelfontsize=10pt}
        \newchessgame
        \chessboard[
            setfen={8/2b1k3/7p/p1p1p1pP/PpP1P1P1/1P1BK3/8/8 w - - 0 1}
        ]
   \end{minipage}
   \caption*{Stalemate, insufficient material, and dead position.}\label{fig:stalemate-insufficient-material-dead-position}
\end{figure}

\noindent Players can also resign at any time, conceding victory to the opponent. Also, if a player runs out of time in a timed game, they lose unless the opponent does not have enough material to checkmate, in which case the game is drawn.

\subsection*{Notation}

Notation is important in chess to record moves and analyze games.

\subsubsection*{Algebraic notation}

In addition to the algebraic notation of the squares mentioned in the chessboard definition, each piece is identified by an uppercase letter, which may vary across different languages, as shown in the following table:

\begin{table}[H]
    \centering
    \begin{tabular}{|c|c|c|}
        \hline
        Piece & English Notation & Spanish Notation \\ \hline
        Pawn           & \textit{P}               & \textit{P} (peón)         \\ \hline
        Rook           & \textit{R}               & \textit{T} (torre)        \\ \hline
        Knight         & \textit{N}               & \textit{C} (caballo)      \\ \hline
        Bishop         & \textit{B}               & \textit{A} (alfil)        \\ \hline
        Queen          & \textit{Q}               & \textit{D} (dama)         \\ \hline
        King           & \textit{K}               & \textit{R} (rey)          \\ \hline
    \end{tabular}
    \caption*{Chess piece notation in English and Spanish.}
\end{table}

\paragraph{Normal moves (not captures nor promoting)}
are written using the piece uppercase letter plus the coordinate of destination. In the case of pawns, it can be written only with the coordinate of destination. In~\cref{fig:pawn-a6}, the pawn's movement is written as $Pa6$ or directly as $a6$.

\vspace{1em}

\begin{figure}
    \centering
    \newchessgame
    \chessboard[
        setfen={r1bqkbnr/pppp1ppp/2n5/1B2p3/4P3/5N2/PPPP1PPP/RNBQKB1R b KQkq - 0 1},
        pgfstyle=straightmove, color=blue,
        markmoves={a7-a6},
        arrow=to
    ]
    \caption{Pawn goes to $a6$.}\label{fig:pawn-a6}
\end{figure}

\paragraph{Captures}
are written with an \texttt{x} between the piece uppercase letter and coordinate of destination or the captured piece coordinate. In the case of pawns, it can be written with the column letter of the pawn that captures the piece. Also, if two pieces of the same type can capture the same piece, the piece's column or row letter is added to indicate which piece is moving.

\vspace{1em}

\begin{figure}
    \centering
    \newchessgame
    \chessboard[
        setfen={r1bqkbnr/1ppp1ppp/p1n5/1B2p3/4P3/5N2/PPPP1PPP/RNBQKB1R w KQkq - 0 1},
        pgfstyle=straightmove, color=red,
        markmoves={b5-c6},
        arrow=to
    ]
    \caption{Bishop captures knight.}\label{fig:bishop-captures-knight}
\end{figure}

\noindent In~\cref{fig:bishop-captures-knight}, the white bishop capturing the black knight is written as $Bxc6$. If it were black's turn, the pawn on $a6$ could capture the white bishop, and it would be written as $Pxb5$ or simply $axb5$, indicating the pawn's column.

\paragraph{Pawn promotion}
is written as the pawn's movement to the last row, followed by the piece to which it is promoted. In~\cref{fig:pawn-captures-rook}, white pawn capturing and promoting on $a8$ to a queen is written as $bxa8Q$ or $bxa8=Q$.

\begin{figure}
    \begin{minipage}{0.45\textwidth}
        \centering
        \newchessgame
        \chessboard[
            setfen={r7/1Pp5/2P3p1/8/6pb/4p1kB/4P1p1/6K1 w - - 0 1},
            pgfstyle=straightmove, color=blue,
            markmoves={b7-a8},
            arrow=to
        ]
    \end{minipage}
    \hspace{0.1em}
    \begin{minipage}{0.45\textwidth}
        \centering
        \newchessgame
        \chessboard[
            setfen={Q7/2p5/2P3p1/8/6pb/4p1kB/4P1p1/6K1 b - - 0 1}
        ]
    \end{minipage}
    \caption{Pawn captures rook and promotes to queen.}\label{fig:pawn-captures-rook}
\end{figure}

\paragraph{Castling}\label{sec:castling}
depending on whether it is on the king's side or the queen's side, it is written as $0-0$ and $0-0-0$, respectively.

\paragraph{Check and checkmate}
are written by adding a $+$ sign for check or $++$ for checkmate, respectively. In~\cref{fig:black-queen-checkmates}, black queen movement checkmates and it is written as $Qh4++$.

\begin{figure}
    \centering
    \newchessgame
    \chessboard[
        setfen={rnb1kbnr/pppp1ppp/8/4p3/6Pq/5P2/PPPPP2P/RNBQKBNR w KQkq - 0 1},
        pgfstyle=straightmove, color=blue,
        markmoves={d8-h4},
        arrow=to
    ]
    \caption{Black queen checkmates.}\label{fig:black-queen-checkmates}
\end{figure}

\paragraph{The end of game notation} indicates the result of the game. It is typically written as:

\begin{itemize}[itemsep=1pt]
    \item \textit{1-0}: White wins.
    \item \textit{0-1}: Black wins.
    \item \textit{1/2-1/2}: The game ends in a draw.
\end{itemize}

\subsubsection*{Forsyth–Edwards notation (FEN)}

This notation describes a specific position on a chessboard. It includes 6 fields separated by spaces: the piece placement, whose turn it is to move, castling availability, \textit{en passant} target square, halfmove clock, and fullmove number.
For example, the FEN for the starting position is:
\begin{verbatim}
rnbqkb1r/pppppppp/8/8/8/8/PPPPPPPP/RNBQKBNR w KQkq - 0 1
\end{verbatim}

\noindent The fields of the FEN for the starting position are:

\begin{itemize}[itemsep=1pt]
    \item \texttt{rnbqkb1r/pppppppp/8/8/8/8/PPPPPPPP/RNBQKBNR}: Piece placement on the board, from rank 8 to rank 1. Uppercase letters represent white pieces, lowercase letters represent black pieces. Numbers indicate consecutive empty squares.
    \item \texttt{w}: Indicates which side is to move (\texttt{w} for white, \texttt{b} for black).
    \item \texttt{KQkq}: Castling availability. \texttt{K} (White kingside), \texttt{Q} (White queenside), \texttt{k} (Black kingside), \texttt{q} (Black queenside). If no castling is available, a hyphen (\texttt{-}) is used.
    \item \texttt{-}: \textit{En passant} target square. If there is no \textit{en passant} possibility, a hyphen is used.
    \item \texttt{0}: Halfmove clock, counting the number of halfmoves since the last pawn move or capture (for the fifty-move rule).
    \item \texttt{1}: Fullmove number, incremented after each black move.
\end{itemize}

\noindent Keep in mind this notation is very important for chess engines, including \textit{AlphaDeepChess}, because it provides all the necessary information to reconstruct any chess position, including all the mentioned variables above on the current board without needing to replay all the moves from the beginning of the game. This makes it highly efficient for testing, analysis, and interoperability with other chess software.

\subsubsection*{Portable game notation (PGN)}

This notation is a widely used text-based format for recording chess games. Its primary purpose is to store complete game records, making it ideal for building large databases of chess games for study, analysis, and sharing. These PGN files not only include the sequence of moves, but also a header section with metadata such as the name of the event, site, date of play, color and name of each player, result and possible comments and variations. For example, PGN for a game could look like this:

\vspace{1em}

\begin{lstlisting}[basicstyle=\ttfamily\small, captionpos=b, breaklines=true, frame=single, caption={Example of a PGN file}, label={lst:pgn-example}]
[Event "Rated blitz game"]
[Site "https://lichess.org/8USBsBgk"]
[Date "2025.05.19"]
[White "Zimbabwean_chessbot"]
[Black "AlphaDeepChess"]
[Result "0-1"]
[GameId "8USBsBgk"]
[UTCDate "2025.05.19"]
[UTCTime "05:41:30"]
[WhiteElo "2865"]
[BlackElo "1904"]
[WhiteRatingDiff "-32"]
[BlackRatingDiff "+12"]
[WhiteTitle "BOT"]
[BlackTitle "BOT"]
[Variant "Standard"]
[TimeControl "180+2"]
[ECO "B12"]
[Opening "Caro-Kann Defense: Advance Variation, Short Variation"]
[Termination "Time forfeit"]

1. e4 c6 2. d4 d5 3. e5 Bf5 4. Nf3 e6 5. Be2 Nd7 6. O-O h6 7. a4 Ne7 8. c3 a6 9. Nbd2 g5 10. b4 Bg7 11. Nb3 O-O 12. Ra2 f6 13. exf6 Bxf6 14. h4 g4 15. Ne1 h5 16. g3 Qc7 17. Ng2 e5 18. f3 gxf3 19. Bxf3 exd4 20. Bf4 Ne5 21. Nxd4 Nxf3+ 22. Qxf3 Qb6 23. Qxh5 Bxd4+ 24. cxd4 Qxd4+ 25. Raf2 Bg6 26. Qg5 Qxb4 27. Be5 Rxf2 28. Rxf2 Rf8 29. Qh6 Rf6 30. Rxf6 Qb6+ 0-1
\end{lstlisting}

These files are generated both during the testing phases of the chess engine and when playing games on online chess platforms.
